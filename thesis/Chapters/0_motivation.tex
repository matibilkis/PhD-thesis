Motivating this thesis should be quite easy, a friend told me a couple of days ago: just ask ChatGPT. However, I will not do this, and the main reason for that is because I have actually found joy during the writing of my thesis, and I think this is a good starting point. If we find pleasure in doing certain things, like research for example, why would we care in finding protocols to \textit{automatize} them, as done in this thesis? \textit{Ups}: tricky question.

Well, I can think of two reasons. The first one is for the fun in doing so: it has been (and still is) an amazing experience! Yes, definitely fun... but... it does not sound very convincing that my only motivation to do a PhD in quantum machine learning was only for the fun of it, right? %Also, let us just say it: this answer is a bit cliché, \textit{no?}

Thus, we need to think on a second reason. Think, think... come on... four and a half years thinking on this PhD and now nothing... Well, at least we can try to imagine the \textit{shape} of the answer. It should be something like... mmmm.... a bit harder to swallow... but not too much, since we really want to keep the reader up to (at least) the end of this Section. Exactly! that's the concept I need: ``Sections!''

While Sections help us structure things (and I can assure you, my Dear Reader that we \textit{will use them} later in the thesis), they do also place a distance between things. For us scientits, such distance is necessary --- and often essential --- in order to focus on an specific research topic. But more often than not, scientific communities (and we can well generalize this to the \textit{human knowledge}) fall into an abuse of \textit{knowledge compartmentalization}. And this sounds like a more satisfying motivation for why doing this PhD: work on connecting Science back. In turn, this thesis brings two fields that ---back to 2018, when my PhD began--- were quite far away from each other. Moreover, and as it happens in many situations when bringing two different worlds toghether, many beautiful things come up. While I will not play the card that quantum AI could lead to a major breakthrough ---I could sell it, but I don't buy it---  I do think that research directions that focus on intersecting different communities with each other are \textit{healthy} for science, and this is some kind of \textit{enough motivation} to me.

So what about \textit{quantum}? and what about \textit{AI}? Falling (perhaphs not too much!) in contradiction with the previous paragraph, let's go by parts.

Quantum theory is a route to understand physical reality, and I want to put emphasis on its fundamental aspects. Those have to do with describing reality by means of quantum superpositions and quantum correlations. Moreover, the theory is intrinsically stochastic in nature. Yet the physical reality that we inhabit has --- a priori --- nothing to do with this wierd phenomena of superpositions and non-local correlations. Furthermore, classical physics is intrinsically deterministic, in the sense that with precise knowledge of a given system, we can perfectly predict its behaviour. Some authors describe this gap as a desert of ignorance occupying many decimals in the scale. Then, how is our classical $\&$ deterministic perseption of reality reconciled with quantum theory? We simply do not know the answer yet, and this is perhaps one of the biggest misteries that we currently need to deal with.

On the other hand we have artificial intelligence (AI), and a huge (classical) computing power available ---at least to some--- right now. While ChatGPT and all the advent of AI is truly impressive, we are certainly not living in an Asimov's tale, and many questions remain open on how can current AIs become \textit{truly intelligent} as we humans are (if there even exists a clear definition for such concept). As of now, the possibility of an AI having the scientific \textit{aha!}s that we scientists are often illuminated with (call it \textit{intuition}) sounds like a fantasy. Perhaphs this will never happen, and our human condition hindered us to completely understand the logic behind our reasoning (be it concious or unconcious).

However, a PhD thesis consists in leaving almost all of the unresolved misteries of science, technology and humanity aside for a while, but focusing on only \textit{some}. And I guess this is what motivates the current thesis: on the one hand there is clear evidence that we can control quantum systems up to an unprecedented level\footnote{By this, we mean preparing and storing a system in a quantum superposition for a sufficiently long time.}. On the other hand, while current AIs might be a game-changer in a volatile society like ours, they do not represent a change-of-paradigm on how intelligence is understood, as they do not even get close to it. This might sound polemic, I know, since for example ChatGPT has recently passed Scott Aaranson's quantum-mechics exam with a 7/10 score. However, I do refer to intelligence as something beyond passing an exam. Again, I would be truly impressed (and of course chances are that I am wrong) that an AI had an \textit{aha!} all of a sudden, and came up with a new theory of knowledge on its own, \textit{i.e.} without human intervention. AI should be understood as a tool, and not as a stand-alone system.

\bigskip
In this thesis, we pay special attention to the currently available quantum technologies, by considering three paradigmatic devices. First we will study continuous-variable quantum receivers, which are aimed to decode information out of a quantum state of light. Secondly, we focus on NISQ computers, which are relatively small and extremely noisy version of quantum computers, but have already been able to provide a quantum advantage with respect to any classical computer paradigm (though for a very specific task). Thirdly, we will consider the case of continuous-time quantum sensors, where a quantum system stored in an optical cavity is being monitored, which provides an effective way to infer information about its environment.

In this thesis we tackle each of the aforementioned devices ---also called quantum environments--- from the machine-learning perspective. This is done by introducing an agent equipped with a varying level of awareness on the quantum environment that is faced to.

The phyisical systems that we consider, along with the methods and machine-learning algorithms, are
widely explained in the \textit{Preliminaries} Chapter. This (considerably long) chapter serves as a reference for the subsequent ones. Thus, the reader can straightfowardly begin from any Chapter, and consult the \textit{Preliminaries} when necessary (appropiate reference are included).

The learning journey begins \textit{in the darnkness}, where the agent is asked to optimally callibrate a quantum receiver without any knowledge of the setting at hand. Here, the only information given to the agent is a binary reward (a bit, valued either 0 or 1), which stands for the correctness of the guess made for the underlying signal that is aimed to be decoded.

We continue \textit{in the twilight}, with an agent facing the problem of building noisy quantum circuits, in the context of NISQ computing. Here, the solutions the agent is asked to reach are even unknown to us, and we can only aid her by providing specifically-tailored circuit compression rules. To this end, we introduce the VAns algorithm, which sequentially modifies the quantum circuit by following a variable-ansatz structure. Our method is thoroughly tested in paradigmatic quantum machine learning applications, ranging from VQE to quantum autoencoder tasks.

Finally, we \textit{learn during daylight}, where a fully model-aware agent is asked to tackle statistical inference problems. Here, our attention is put on continuously-monitored quantum systems, where the performance of sequential discrimination protocols is studied. Here, we analytically study the evolution of the log-likelihood ratio, by providing insight on error rates and stopping time probability distribution for a sequential test. In addition, the case of parameter estimation is considered, and a maximum-likelihood estimation is carried out by an automatic-differentiation method; such recurrent structure allows us to infer parameters of external signals out of the measurement signal.

The main contribution of this thesis is that of `` bringing the quantum scientist closer to the AIs, in such a way that they can collaborate with each other". This thesis, as most of the scientific research, provides a (small) step further towards the constitution of such collaboration. Having this in mind, the reader shall not understand each of the Chapters as a closed-one, but rather as an invitation to continue writing it. In turn, our results do not constitute a milestone in quantum AI, but we rather see them as blueprints towards more ellaborate quantum AI framework.
