A measurement is deemed gaussian if \textit{(i)} it preserves the Gaussian character of the state upon post-conditioning, and \textit{(ii)} when measuring a Gaussian state, the outcome probability distribution is Gaussian.
For instance, counting excitations correspond to projections over Fock states: recall that $\sum_{n=0}^\infty \proj{n} = \mathbb{I}$), and are obviously non-gaussian since number states (asside from vacuum) are not; moreover, the underlying distribution is Poissonian. Nevertheless, it is experimentally possible (though expensive) to resolve excitations in quantum optics laboratories, thorugh an apparatus known as \textit{photon-counter}~\cite{Photoncounter}. In this regard, one may be interested in only resolving between ```no photons'' or ```one or more photons'', an apparatus known as \textit{on/off photodetector}, which correspond to jut partitioning the Hilbert space differently, \textit{e.g.} $\mathcal{M}_{on/off} = \llaves{\proj{0}, \mathbb{I}-\proj{0}}$, where the second measurement operator is non-gaussian.

Le us know restrict to Gaussian projections. In particular, we have mentioned homodyne detection corresponds to projecting over a quadrature direction in phase space, as defined by the operator $\hat{q}_\theta = \cos\theta \hat{q}+ \sin \theta \hat{p}$, where $\hat{q}_\theta \ket{q_\theta} = q_\theta \ket{q_\theta}$ and $\mathbb{I} = \int_\mathbb{R} \proj{q_\theta} dq_\theta$, and $p(q_\theta) = \tr{\proj{q_\theta} \rho}$.
Such measurement can be realized, for instance, by the so-called \textit{balanced} homodyne detection. This consists in mixing $\rho$ with a local oscillator ($\ket{\alpha}$ with $|\alpha|\gg1$) by a balanced beam-splitter ($\theta = \frac{\pi}{4}$ and $\psi=0$ in Eq~\ref{eq:colhonbs}), and substracting the detected intensities at the two outputs of the beam-splitter (each measured by a photodetector). An alternative way to realize a homodyne detection is the \textit{direct} homodyne detection, which consists on mixing the incoming signal with a local oscillator by a low-reflectivity BS (which transmits most of the original signal, adding only a small amount of the local oscillator) and measuring the intensity of the reflected port.

On the other hand, we can consider heterodyne measurements, which in accordance to Eq.~\eqref{eq:resocoh} are projective measurements on coherent state of varying intenstity and phase. They can be realized by combining the state $\rho$ with a vacuum into a balanced beam splitter, and homodying in the direction of $\hat{q}$ and $\hat{p}$ (\textit{e.g.} $\theta = 0 $ and $\theta = \frac{\pi}{2}$ in Eq.~\ref{eq:passive1} respectively) in the paragraph of above); in this regard heterodyne measurements correspond to joint measurements of position and momenta, and its probability measurement outcome is $p(\alpha) = \langle\alpha|\rho|\alpha\rangle\pi^{-1}$.

Such a mechanism of implementing a measurement by homodying and having access to ancillas can actually be generalized to any Gaussian operation via Gaussian ancillas, Gaussian unitary interactions and homodyning/discarding the reduced state of the ancilla system~\cite{Cirac2002Characterization}.
