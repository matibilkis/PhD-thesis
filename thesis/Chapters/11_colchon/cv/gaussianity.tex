We define Gaussian quantum states as thermal states of a quadratic hamiltonians
\eq{rhog}{\rho_G = \frac{e^{-\beta \hat{H}}}{\tr{e^{-\beta \hat{H}}}} \space,   \; \; \beta\in\mathbb{R}^+,}
\eq{hamiGauss}{\hat{H} = \frac{\rop^T H \rop}{2} + \rop^T \rvec,}
where $\rvec$ is a constant vector in phase space, and $H$ is the \textit{Hamiltonian matrix}, which is positive-defined and symmetric. Note this definition includes the limit $\beta \rightarrow \infty$, which corresponds to the case of pure states.

Such construction allows us to find the so-called normal mode form of $\rho_G$, in which the system behaves as a set of $n$ decoupled harmonic oscillators. In particular, studying the transformation that takes $\rho_G$ into its normal form is instructive, and we will comment on it nextly. %when studying the structure of gaussian states.
%
% Let us understand the structure of $\hat{H}$ in detail.
There are at least two important reasons to study Gaussian states. Firstly, many problems become analytically tractable when dealing with such a class. Secondly, manipulating Gaussian states is experimentally feasible. Moreover, light (\textit{e.g}) is a natural candidate for quantum communication scenarios, and coherent states --- a particular type of Gaussian states --- describe very well the behaviour of lasers.

We note that, up to a constant shift in the energy landscape, $\hat{H}$ is equivalent to
\eq{h1}{\hat{H}' = \frac{(\rop - \rbar)^T H (\rop - \rbar)}{2},  \spacee \rbar = - H^{-1} \rvec.}
Note that $H^{-1}$ does always exists since $H>0$. It follows that
\eq{hp}{\hat{H}' = \frac{1}{2} \weyl{-\rbar} \rop^T H \rop \weyl{\rbar},}
we can thus focus on the non-displaced hamiltonian $\rop^T H \rop$. In the Heisenberg picture, the evolution of canonical operators corresponds to $\dot{\rop}_j = \ii \Comm{\hat{H}}{\rop_j}$, and by using the CCR, it follows that $\dot{\rop} = \Omega H \rop$. Thus, we conclude that $\rop(t) = e^{\Omega H t} \rop(0)$.
Since any system transformation should preserve the CCR, it follows that
\eq{pres}{\ii \Omega = \Comm{\rop(t)}{\rop^T(t)} =  e^{\Omega H } \Comm{\rop(0)}{\rop(0)^T} (e^{\Omega H })^T  = \ii e^{\Omega H } \Omega (e^{\Omega H })^T,}
and thus
\eq{symp}{\Omega = S \Omega S^T, \spacee S:= e^{\Omega H t}.}
The last line is the condition for $S$ to belong to the real symplectic group $\symp$, which is the reason for calling $\Omega$ the symplectic form. If we now explicit the unitary character of canonical operators' time-evolution, we obtain
\eq{00}{\dot{\rop} = \hat{U}^\dagger(t) \rop(0) \hat{U}(t) = e^{- \ii \hat{H} t} \rop(0) e^{\ii \hat{H} t} = e^{\Omega H t} \rop(0) = S \rop(0),}
we see that there is a clear correspondence between unitary operators and symplectic transformations, and in fact $\symp \cap SO(2n)$ is a isomorphic to $U(n)$. Thus, there is a correspondence between symplectic transformations $S=e^{\Omega H}$ and unitary operators generated by second-order hamiltonians, $\sh=e^{\frac{\ii \rop^T H \rop}{2}}$:
\eq{sympo}{\sh \rop \sh^\dagger = S \rop.}
In particular, said symplectic transformation $S$ can be written in its singular value decomposition as
\eq{svdS}{S = O_1 Z O_2,}
where $O_1, O_2 \in \symp\cap SO(2n)$ are orthogonal symplectic transformations and $Z = \bigoplus_{j=1}^{n}\begin{pmatrix}z_j&0\\0&z_j^{-1}\end{pmatrix}$ stands for the \textit{squeezing} transformation. The energy of the system ($\rop^T\rop$) is invariant under the action of $O_i$, and thus we refer to those transformations as \textit{passive} ones. Importantly, the isomorphism with the unitary group guarantees that such symplectic transformations can be constructed out from \textit{phase-shifters} and \textit{beam-splitters}~\cite{Reck1994Experimental,Weedbrook2012Gaussian}.

In particular, phase-shifters $R_\phi$ and beam-splitters $B_{\theta,\psi}$
have the following matrix representations
\eq{passive1}{R_\phi = \begin{pmatrix}\cos\phi&\sin\phi\\ -\sin\phi & \cos\phi\\ \end{pmatrix},\spacee B_{\theta,\psi}=\begin{pmatrix}\cos\theta&e^{i\psi}\sin\theta\\ -e^{-\ii \psi}\sin\theta& \cos\theta\\ \end{pmatrix}\otimes \mathbb{I}_2.}
In passing, we note that if we move the anhilation and creation operators, the action of phase-shifters is that of adding a phase $e^{\ii \phi}$, whereas beam-splitters mix two modes $a_1$ and $a_2$ (which correspond to the input modes):
\begin{align}\label{eq:colhonbs}
B_{\theta,\psi}^\dagger \hat{a}_1 B_{\theta,\psi} &= \cos\theta \hat{a}_1 + e^{\ii\psi}\sin\theta \hat{a}_2 \\
B_{\theta,\psi}^\dagger \hat{a}_2 B_{\theta,\psi} &=  -e^{-\ii\psi}\sin\theta \hat{a}_2 +\cos\theta \hat{a}_1,
\end{align}
where the added phase $\psi$ can, for instance, be corrected via a proper phase-shift.

To sum up, any simplectic transformation can be implemented out of squeezing operations and interferometers (\textit{e.g.} a combination of phase-shifters and beam-splitters, which can in turn mimic any passive simplectic transformation.%~\cite{Reck1994Experimental}).

Let us now focus on the structure of the Hamiltonian matrix $H$. In particular, we recall that given $M\in\mathbb{R}^{2n\times2n}$, with $M>0$, there exists a transformation $S\in\symp$ taking $M$ into its normal form~\cite{Williamson1936Algebraic}, \textit{e.g.} $SMS^T = D$  with $D = \bigoplus_j d_j\mathbb{I}_2$ %\text{diag}\big(d_1,d_1...d_n,d_n\big)
containing the symplectic eigenvalues $\llaves{d_j}_{j=1}^n$; recall that such transformation can be challenging to find~\cite{pereira2021symplectic, Pirandola2009correlation}. In this regard, $S$ is related to the transformation
$L$ that puts $\ii \Omega M$ into its diagonal form, \textit{e.g.}
$L \Omega M L^{-1}$ is diagonal, with $S = (L^{-1} \bar{U})^\dagger$, and the symplectic eigenvalues of $M$ are the (absolute values of) the eigenvalues of $\ii \Omega M$, which come into pairs $\llaves{\pm d_j}_{j=1}^n$.
We can readily apply such decomposition to our quadratic Hamiltonian matrix $H$:
\equ{H = S^T \big( \bigoplus_{j=1}^n \omega_j \id_2 \big) S,}
where by denoting the symplectic eigenvalues by $\omega_j$ we highlight its role as frequencies of $n$ non-interacting free modes.
By inserting such normal mode decomposition into $\hat{H}$:
\begin{align}
\hat{H} &= \frac{1}{2} \rop^T H \rop \\
&= \frac{1}{2} \rop^T S^T \big(\bigoplus_{j=1}^n \omega_j \id^{(j)}_2 \big)S \rop
\end{align}
We now use that the symplectic transformation $S$ has a unitary representation $\sh$:
\begin{align}
\hat{H} &= \frac{1}{2} \rop^\dagger H \rop \\
&= \frac{1}{2} \hat{S} \rop^T \big(\bigoplus_{j=1}^n \omega_j \id^{(j)}_2 \big) \rop \hat{S}^\dagger
\end{align}
Upon defining $\hoj := \frac{\omega_j}{2}(q_j^2 + p_j^2)$, we note that any second-order Hamiltonian with zero displacement is equivalent (up to a unitary transformation) to a set of decoupled oscillators. The decomposition of the original quadratic hamiltonian reads
\begin{align}
\hat{H} &= \frac{1}{2} (\rop - \rbar)^T H (\rop - \rbar) \\
&= \weyl{-\rbar} \hat{S} \big( \sum_{j=1}^n \hoj\big) \hat{S}^\dagger \weyl{\rbar},
\end{align}
which implies that gaussian states have the following structure:
\begin{align}\label{eq:1gaussrho}
\rho_G &= \frac{\weyl{-\rbar} \hat{S} \big( \bigotimes_{j=1}^n e^{- \beta \hoj} \big) \hat{S}^\dagger \weyl{\rbar} }{ \prod_j\tr{e^{-\beta \hoj} }} \\
&= \mathcal{N}_\beta \weyl{-\rbar} \hat{S}^\dagger \big( \bigotimes_{j=1}^n \sum_{k_j=0}^\infty e^{- \beta k_j \omega_j} \proj{k_j} \big) \hat{S} \weyl{\rbar},
\end{align}
where in the second line we have expanded $\hoj$ in the Fock basis $\llaves{\proj{k_j}}$ of each mode $j$, and defined $\mathcal{N}_\beta  := \prod_j (1-e^{-\beta \omega_j})$. We observe that for $\beta \rightarrow \infty$ a pure state is obtained, since only vacuum terms survive. Hence, pure gaussian states are generated by applying gaussian unitary operations to the n-mode vacuum state. Quite remarkably, either Gaussian states have rank 1 (pure states), or infinite rank (mixed states).

We thus conclude that Gaussian states are obtained from orthogonal symplectic transformations (and displacements) applied to a set of non-interacting harmonic oscillators. In particular, the action of such transformations is lineal in phase space, a fact that becomes clear when inspecting its action on the statistical moments.
