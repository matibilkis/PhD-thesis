Another parametrization of Gaussian states can be obtained, by departing from Eq.~\ref{eq:1gaussrho}; in particular we note that the first two moments of $\rho_G$ encode all its information, as
\begin{align}\label{eq:1moms}
 \rbar&:=\tr{\rho_G \rop}   \\
\cov&:= \tr{ \llaves{\rop - \rbar, (\rop - \rbar)^\dagger} \rho} = S \big(\bigoplus_j \nu_j \id_2^{(j)} \big)S^T,\nonumber
\end{align}
where $\nu_j = \dfrac{1+e^{-\beta \omega_j}}{1-e^{-\beta \omega_j}} $ stand for the symplectic eigenvalues of $\cov$\footnote{Note that for the single-mode case, the covariance reads $\cov_{nm} = \expect{\Delta \rop_n \Delta \rop_m} + \expect{\Delta \rop_m \Delta \rop_n}$, with $\Delta \rop_i = \rop_i - \expect{\rop_i}$, and $\rop = (q, p)$.}.
All the information about the Gaussian state is encoded in both the expectation value and covariance. On the one hand, the displacement value $\bm{\bar{r}}$ stand for the first statistical moment of the Gaussian state. On the other hand, the symplectic transformation $S$ (associated to the gaussian unitary $\sh$), as well as the symplectic eigenvalues of $H$ can be obtained through $\cov$. In this regard, we note that the spectrum of the Guassian state can be obtained from the covariance matrix, and thus correlation measures between the modes, as well as the states' purities, can be inferred from the second statistical moment.

%%% robertson- schrodinger
Importantly, any covariance matrix $\cov$ of a quantum state should satisfy the Robertson-Schrödinger uncertainty relation
\eq{uncer}{\cov + \ii \Omega \geq 0,}
which is a direct consequence of the CCR and the positivity of the quantum state $\rho$.
The singular value decomposition of symplectic transformations in~\eqref{eq:svdS} turns useful to understand the structure of the covariance matrix. For example, if we restrict to pure Gaussian states, then it follows that
\begin{align}\label{eq:pure_cov_gauss}
\cov = O D O^T,
\end{align}%single-mode Gaussian states, it follows that $\cov = \nu O D O^T$, with $D=\text{diag}(d, \frac{1}{d})$, whereas if we consider $n$can be inserted into the expression
where $D = \bigoplus_{j=1}^n\begin{pmatrix}d_j&0\\0&\frac{1}{d_j}\\\end{pmatrix}, \; d_j>0$ and $O$ stands for a passive, orthogonal transformation. Moreover, if we restrict to single-mode systems, we observe that covariance matrices are generated by squeezing and rotating the identity; this is aligned with the fact that all pure gaussian states can be obtained through symplectic tranformations applied to the vacuum state $\ket{0}^{\otimes n}$.

While preparation of pure Gaussian states has such intuitive interpretation, this picture can be complemented with a collection of quasiprobability distributions that depict the quantum state in phase space which we now turn to present.
