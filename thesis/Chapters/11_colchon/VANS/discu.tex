In this Section we have discussed about NISQ computing, and strategies to train parametrized quantum circuits. We have introduced the quantum machine learning gargeon, namely a cost-function, an ansatz and an optimizer. Then, we have discussed about trainability issues that appear due to the fact that cost-function derivatives concentrates around its mean-value (\textit{e.g.} zero); such value of the gradient does not correspond to a local minima, but rather to a pleateau, known as barren pleateau. Such landscape represents an important issue for optimizers, since finding cost-minimizing directions is forbidenly resource-consuming. Moreover, we note that the barren-pleateu phenomena has also been observed in gradient-free optimizers~\cite{arrasmith2020effect}. Under the vast literature studying this trainability issue, we particularly emphasized that neither strategies that mitigate randomness nor entanglement in ansatzes~\cite{verdon2019learning,volkoff2021large,skolik2020layerwise,grant2019initialization,pesah2020absence,zhang2020toward,bharti2020quantum,cerezo2020variational}, which cause the appearence of barren pleateaus will be effective in mitigating a different source of landscape-flatness, called noise-induced barren pleateau.

In this regard, it is widely accepted that designing smart ansatzes which prevent altogether barren plateaus is one of the most promising applications. In Chapter~\ref{chapter:VANS} we move a step-forward in this direction, providing a method to train quantum circuits on both parameter and structure-wise.
%
% With this, we will now turn to discuss the formalism of continuous-variable quantum systems, which will be used in ~\ref{ch:chapter}
 %
 %
 % While several strategies have been developed to mitigate the randomness or entanglement in ansatzes prone to barren plateaus,
