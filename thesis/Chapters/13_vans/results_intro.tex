In this Section, we present our results obtained from simulating VAns to solve paradigmatic problems in condensed matter, quantum chemistry, quantum autoencoding and quantum compiling problems in ideal conditions,\textit{e.g.} without considering any noise-model of the quantum computer.

We first use VAns in the Variational Quantum Eigensolver (VQE) algorithm~\cite{peruzzo2014variational} to obtain the ground state of the Transverse Field Ising model (TFIM), the XXZ Heisenberg spin model, and the $H_2$ and $H_4$ molecules. We then apply VAns to a quantum autoencoder~\cite{romero2017quantum} task for data compression. We then move to use VAns to compile a Quantum Fourier Transform unitary in systems up to 10 qubits. In all these cases, we have perform the simulations under the unrealistic scenario in which neither shot-noise nor harware noise were considered; this will be the matter of further sections. %Finally, we benchmark the performance of VAns under the presence of noisy channels, which are unavoidable in quantum hardware, and demonstrate that it successfully finds cost-minimizing circuits on a wide range of noise-strength levels.

The simulations presented here were performed using Tensorflow Quantum~\cite{broughton2020tensorflow}. Adam~\cite{kingma2015adam} and qFactor~\cite{qFactor} were employed to optimize the continuous parameters $\thv$, \textit{e.g.} \texttt{Opt}$_C$ in the pseudo-code previously presented. While VANs is an stochastic algorithm, and the number of iterations required until meeting a convergence criteria should in principle be a random variable, we remark that all the results shown in this Section were obtained from \textit{a single instance} of the algorithm (for each of the problems considered). While a statistical analysis should in principle be carried out in order to analyze VAns' average performance, the latter fact intuitively highlights the way in which VAns is able to exploit computational resources.

The dictionary $\mathcal{D}$ of gates used consisted on block of single single-qubit and two-qubit gates that can resolve to identity. Each is composed by single-qubit rotations around $x$ and $z$ axis, and CNOTs gates between all qubits present in the circuit. As already mentioned, we have assumed no connectivity constraints under the quantum circuits under consideration. In the following examples, VAns was initialized to either a separable ansatz or an $L$-HEA, with $L<3$ (recall that in Fig.~\ref{fig:FANSATZ} we showed the separable and 2-HEA circuits).
