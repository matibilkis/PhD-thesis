Our journey through the learning in the quantum begins, literally, by trial and error. We focus on quantum state discrimination (see Sec.~\ref{ssec:1_qdisc}), where we want to tell which is the physical state of our system out of measurement outcomes. In this Chapter, learning occurs from the darkness, and we ask how the discrimination error can be minimized by using only binary reward signals, accounting for the correctness of the guess.

To this end, we will focus in optical systems, and ask a model-free reinforcement-learning agent (see Sec.~\ref{sec:1_rl}) to control an optical table; by departing from complete ignorance about the setting, we require the agent to achieve near-optimal discrimination performance. Our approach is crucially focused in experimental scenarios, where each repetition of the experiment counts, and thus exploiting every measurement outcome is required.

Studying this setting is particularly motivated by long-distance classical communication over quantum channels. Here, classical information is encoded into a quantum state, which is sent by a quantum channel to a receiver. Once arrived, the original information needs be decoded as accurately as possible, and thus a non-trivial optimization over quantum measurements arises. For instance, in ground-to-satellite communication, optical signals are sent through the atmosphere, which can degrade signals' intensity to the point that quantum distinguishability effects become relevant. A quantum receiver is thereby used, which decodes the original information sent from free space. %Here, we can readily identify a tradeoff: while the measurement that best distinguishes between the states is a projection over the candidates difference (\textit{e.g.} Eq.~\ref{eq:1_qdisc_helstom}), it might be very difficult to implement experimentally (and even to install on the satellite).

We consider one of the most standard sources of light found in optical laboratories, which are lasers. The quantum-mechanical behaviour of such systems is described by coherent states~\cite{Glauber1963Coherent} $\ket{\alpha}$, that belong to the set of Gaussian states (see ~\ref{ssec:intro_cv_gaussianinfo}). Specifically, we cast the discrimination of two electromagnetic signals with opposite phases, described by two coherent states of the field, $\ket{\alpha_k}$, with $\alpha^{(k)} = (-1)^{k}\alpha$, whose energy is proportional to $|\alpha|^{2}$.
When the energy of the signals approaches zero, \textit{i.e.}, $|\alpha|^{2}\ll1$ (or when losses are present in the transmission channel), quantum effects become evident and it becomes impossible to discriminate between them perfectly. In particular, the ultimate bound given by quantum mechanics imposes bounds to the distinguishability of states, as discussed in Sec.~\ref{ssec:1_qdisc}. For the discrimination between two coherent states, the Helstrom bound (\textit{e.g.} Eq.~\eqref{eq:helstrom_pure}) reads:
\begin{align}\label{eq:hel_coh}
P_{s}^{(hel)}(\alpha)=\max_{\mathcal{M}}P_{s}(\alpha,\mathcal{M})=\frac12\left(1+\sqrt{1-e^{-4\abs{\alpha}^{2}}}\right),
\end{align}
where we recall that the overlap between the two states is $|\braket{-\alpha}{\alpha}|^2 = e^{-4|\alpha|^2}$. Thus, as the intensity of the original signals to zero, the success probability gets closer to that of randomly guessing for the phase of the coherent state.

As discussed in Sec.~\ref{ssec:1_qdisc}, any binary discrimination protocol is described compactly by a POVM, $\mathcal{M}=\llaves{M_{0},M_{1}}$ with $M_{1,2}\geq0$ and $M_{1}+M_{2}=\mathbb{I}$. The probability of obtaining measurement outcome $\hat{k}$ given that hypothesis $k$ was true is given by $p(\hat{k}|\alpha_k) = \tr{M_k \proj{\alpha^{(k)}}}$.
Nevertheless, once outcome $\hat{k}$ has been obtained, \textit{a guess} for $k$ needs to be done, and the best one is --- by definition ---  related to the most likely hypothesis $k\in\llaves{0,1}$, given outcome $\hat{k}$. Thus, the success probability of this setting reads
\begin{align}
P_{s}(\alpha,\mathcal{M})&=\sum_{\hat{k}=0,1}\max_{k=0,1}p(\alpha^{(k)},\hat{k})\\
&=\sum_{\hat{k}=0,1}\max_{k=0,1}p(\hat{k}|\alpha^{(k)})p_{k}.
\end{align}
In particular, the measurement $\mathcal{M}$ that achieves the Helstrom success probability is a superposition on the positive and negative part of an operator $\Lambda = \frac{1}{2}(\proj{\alpha} + \proj{-\alpha})$. As discussed in the exampe of Sec.~\ref{ssec:1_qdisc}, such projection is obtained by a superposition of the original states, which in the coherent-state discrimination problem leads to a projection over cat-like states of the form $\ket{\alpha_0} + \ket{\alpha_1}$~\cite{Osaki1996Derivation}.

While preparation of such states currently constitutes an experimental challenge~\cite{catstates0,catsates1_RL}, an implementation of this projection can be carried out using lineal (\textit{i.e.} Gaussian) optics, photon-detectors and feedback operations. This measurement scheme is known as \textit{the Dolinar receiver}, works in a sequential logic and it is proven to be assymptotically optimal~\cite{Dolinar,Takeoka2005}. Because it only requires lineal optics, photon-detection and feedback operations, this receiver is experimentally appealing, and several proofs of concepts have already been carried out~\cite{Cook2007,Geremia2004,DaSilva2013,DiMario2018}. It constitutes, nevertheless, a non-Gaussian measurement (see Sec.~\ref{ssec:1_cv_measurements}), and as such it presents several implementation challenges; for example, the feedback operation is assumed to act instantaneously, with arbitrarly many operations happening during the measurement process, whereas in practice we are constrained to a limited amount of such operations. Moreover, the presence of noise might alter the performance of the receiver.

However, a strong assumption is needed to implement the Helstrom measurement, or approximate versions of it. Namely, it is (obviously) required that the experimenter knows the quantum states she wants to distinguish. This motivates the current Chapter, where we will focus on the model-free discrmination of coherent  states, by means of a Dolinar-like reveiver. Our motivation emerges from situations where the experimental setting is not well characterized, and where we need to take advantage of \textit{each} device usage (\textit{i.e.} each measurement).

% e calibration of Dolinar-like receivers in a model-free way, and out of several repetitions of the discrimination experiment, will be the matter of this Chapter. In this regard, while experimental implementations issues potentially forbid the optimality of a quantum receiver with respect to the ultimate Helstrom bound, an \textit{imperfect} Dolinar-like receiver can still perform \textit{pretty well}.

In this regard, achieving optimality in the implementation of quantum protocols (such as state discrimination) is certainly a goal, but we shall not dismiss sub-optimal strategies. Several examples supporting this claim can be found in our thesis: we find several scenarios in which sub-optimal strategies become state-of-the-art techniques that can readily be used to tackle a problem at hand. For instance, while the Dolinar receiver is assymptotically optimal when enough \textit{measurement layers} are performed, we will see that already a single layer is sufficient to surpass the best Gaussian receiver (that is, a receiver composed only out of Gaussian elements).

Nevertheless, while the performance of such receivers can prove succesful even in the case of limited resources (for example, a limited number of such measurement layers available), there are certainly more shortcomes that we need to address in order to model realistic scenarios. Among them, there is always the presence of some noise that underlies every experiment. Moreover, in communication scenarios, an unknown source of noise (such as an uncharacterized quantum channel) may alter the transmitted signals in a non-trivial manner. For example, in the case of long-distance ground-to-satellite communications, such channel is the atmosphere, whose action on the coherent-states can be understood as a lossy channel (see Sec.~\ref{ssec:1_cv_channels}) whose attenuation suffers non-trivial variations depending on height or temperature, a situation which turns particularly difficult to model~\cite{Dequal2020,Andrews2005,Usenko2012a,Pirandola2021,Pirandola2021a,Vasylyev2011,Vasylyev2017}.

In this context, our model-free approach to the calibration of resource-limited Dolinar-like receivers finds a solid motivation. Here, an agent needs to find a near-optimal configuration for its initially uncalibrated receiver by trial and error repetitions of the discrimination experiment. At each episode, one of the two possible coherent-states $\ket{\alpha_k}$ is randomly picked and sent to the agent, who needs to make use of its apparatus in order to guess for the signal's label. Here, we stress that no knowledge about the physics of the setting is assumed, but only a \textit{reward} is given to the agent if the bit of information is correctly decoded. As we will show, this approach is robust under the presence of unknown sources of noise, since the agent is no biased towards the noise-free setting and can readily adapt its calibration strategy to the noisy channel at hand.

This Chapter is structured in the following way. We begin in Sec.~\ref{sec:rl_coh_quantum_receivers} by revisting some quantum receivers that are frequently used to discriminate between two coherent-states; in particular we introduce homodyne, kennedy and Dolinar receivers. The sequential structure of the latter is exploited by a dynamic programming optimization in Sec.~\ref{sec:rlcoh_model_aware_approach_intro}. Then, we study how such optimization can be carried out in a model-free way in Sec.~\ref{sec:rl_coh_model_free}.
Our main results are presented in Sec.~\ref{ssec:rlcoh_qlearning}, where a reinforcement-learning example is considered for the idealistic case. Following, we consider how such learning paradigm adapts to a sequence of experimental limitations in Sec.~\ref{ssec:rlcoh_noise}. This Chapter concludes with a discussion and prospects for future extensions of this project in Sec.~\ref{ssec:rlcoh_outlook}.
