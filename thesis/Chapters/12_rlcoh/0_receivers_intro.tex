In this chapter we focus on quantum state discrimination, which was introduced in Sec.~\ref{ssec:1_qdisc}. Here, rather than measuring physical quantities such as position or angular momenta of a physical system, we want to measure \textit{which is} the physical system at hand out of a set of possible labels. %Beyond epistemic consequences of what the \textit{correct} descrpition of a physical system is, \textit{e.g.} what does it mean to correctly classify a quantum state.
This task is a primitive in quantum communications, where information carried out by a quantum state is transmitted from one partie (transmitter) to another (receiver) thorugh a quantum channel. In particular, we focus on a very concrete realization of such a protocol, which is using light. While the usage of light to transmit information is not a novelty --- this thesis can be downloaded in the entire planet thanks to our network of optical fibers --- the manipulation of \textit{quantum} light is by now a mature field in experimental laboratories, and is gradually proving successful for the implementation of several quantum protocols \cite{qprots_cv}.% such as quantum teleportation~\ref{tele}, quantum key distribution ~\ref{qkd_cv}, quantum boson sampling~\ref{bs_cv}, quantum simulation~\ref{qs_cv}.
The experimental implementation of gaussian channels (such as state preparations, operations and measurements) is currently considered \textit{standard} in quantum optical laboratories, whereas some non-gaussian measurements are readily available, such as photon-counters and on/off photon-detectors. While the control of quantum states of light turns to be challenging in the non-gaussian regime, it is also widely accepted that Gaussian states can be manipulated with relatively high accuracy. In this context, non-gaussianity appears as a \textit{resource}, which in turn permits the quantum optical implementation of even fault-tolerant quantum computations, for instance via Guassian operations complemented with the \textit{Kerr} gate, the implementation of the latter stands as one of the biggest experimental challenges. Nevertheless, universal quantum computing needs not to be the ultimate goal on why to study quantum technologies and quantum physics.%, which partially constitutes the spirit of the current chapter (and thesis).

Perhaphs the most standard devices in experimental laboratories are lasers (\textit{e.g.} light amplification by stimulated emission of radiation). A good description of the laser's quantum state is given by quantum coheerent states $\ket{\alpha}$~\cite{ref:molmer_coh}, which turns out to be classical in many senses. For instance, their Wigner function is a Gaussian (\textit{e.g.} they Gaussian states) and thus positive-defined. We will refer to the intensity of the laser as $|\alpha|^2$; in many senses we can refer to the \textit{classical} regime as that one in which $|\alpha|^{2} >> 1$. In particular, the intensity plays a prominent role in coherent states distinguishability. Here, low-intensity coherent states arise as a consequence of lossy-channels (which can be modelled as a gaussian Beam Splitter operation), whose action on a coherent state is that of attenuating its intensity. For instance, in an optical fibre the intensity decreases by $\frac{\%0.01}{km}$~\cite{ref:dec_bs_of}. While the effect of the attenuation might be neglegible on a few kilometers, it certainly plays a role in long-distance communication. Moreover, whereas some scenarios might admit amplifications or repeaters, some others might not, such as communications in outer space. For precisely this reason, the study of coherent-state discrimination in low-intensity regimes becomes not only relevant, but also very interesting.

In there we have derived ultimate bounds for the success probability of of binary-state quantum discrimination. In that setting, either $\rho_0$ or $\rho_1$ stands for the correct description of the quantum system, and the goal is to perform a measurement that maximizes the error probability of mis-identifying the quantum state, in the single-shot setting (\textit{i.e.} single measurement, on a single copy). In particular we obtained an expression for the optimal measurement, \textit{e.g.} the one that saturates the so-called Helstrom bound in Sec.~\ref{sssec:hb}, that turned out to be a projector onto the positive and negative part of $\Delta = \rho_0 - \rho_1$, whose expression migth be obtained through the normal form of $\Lambda$.

In this context, the classicality of coherent states actually hinders some quantumness. It turns out that a highly \textit{non-classical} (\textit{e.g.} non-gaussian) measurement needs to be done in order to optimally distinguish between two coherent states, i.e. $\rho_0 = \proj{\alpha_0}$ and $\rho_1 = \proj{\alpha_1}$ with $\alpha_0 \neq \alpha_1$. As a matter of fact, the Helstrom measurement in this case reduces to a projection onto a cat-like state~\cite{ref:cat_jap}, $\ket{cat}\propto \ket{\alpha_0} + \ket{\alpha_1}$, whose preparation and many efforts are currently been carried out by experimantl teams to even preparing such a state as a proof of principle.

Nevertheless, this is not the end of coherent-state discrimination's story. While projections over cat-like states are currently --- and will probably remain --- beyond the experimental reachability, there exists
a sequential scheme that is proven to be assymptotically optimal. This sequential scheme was proposed by Dolinar ~\cite{dolinar_phd}, and will be the matter of the following sections. While still non-gaussian, this approach does only require Gaussian operations, photon-detectors and a classical feedback. Hence, it constitutes an experimentally appealing approach, since all of those components are readily available in modern quantum optics laboratories. As a matter of fact, experimental proofs of concepts have already been carried out with partial success~\cite{exp_dols}; while the sequential schemme seems to work, different sources of noise degrades it.

Achieving optimality in quantum protocols such as state discrimination is certainly a goal, but we shall not misregard sub-optimal strategies. In the same spirit as in Ch.~\ref{chapter:VANS}, even such strategies might become state-of-the-art methods. Moreover, by restricting the amount of resources available at hand --- for instance forbid Non-Gaussian operations in the state discrimination --- we can certainly get more insight into the problem at hand. Moreover, noise and malfunctioning devices (\textit{e.g.} non-instantaneous feedback in the sequential receiver) essentially forbids optimality and the question raises: \textit{can we get closer to the optimal bound with a resource-limited sequential approach?} This motivates the first part of this Chapter, where we consider Gaussian receivers and in particular study the succss probability of homodyne and heterodyne measurements in ~Sec.\ref{sssec:1_gaussian_receivers}. We then move on to introduce the Kennedy receiver in ~\ref{ssec:2_kennedyreceiver}, which can be thought as the building block of the sequential receiver. The latter are introduced in ~\ref{sssec:3_dolinarreceiver},
and to some extent are $L$ sequential iterations of Kennedy-like receivers.

A \textit{good} sequential receiver, \textit{e.g.} $(L\rightarrow \infty)$ is proven optimal ~\cite{opti_dol}, but here we consider what happens in the case of finite $L$. In \ref{ssec:9_model_aware_approach_intro} we study the success probability of such receivers by using dynammic programming, which was introduced in Sec.~\ref{sssec:15_dp}. In there, we do also consider the optimization lanscape, which turns out to be extremely challenging even in the noise-less scenario. Nonetheless, we are able to compute the success probability up to $L=20$, and get some insight on the optimal strategy, which benefits from the feedback in a non-trivial way.

While able to obtain a value for the success probability of each sequential receiver, there is actually no guarantee that, in practice, the success rate attained will equal that value. Many reasons might cause this shortcome, among them the presence of noise in the apparatus, or more importantly, the action of an unknown channel happening in between the transmitter (which sends the coherent state) and the receiver (which performs the sequential measurement). In practice, this medium is the atmosphere, and the channel turns out to be a lossy channel with an attenuation that varies in a non-trivial way with respect to variables such as height, temperature or even beam waist~\cite{pirandola_atmo}. This motivates our setting, which consists on a model-free approach to the optimization (or calibration) of a sequential receiver. Here, an agent needs find an optimal configuration for this receiver, whose structure has been fixed, through reinforcement learning techniques, which were introduced in Sec.~\ref{ssec:1_rl}. Here we stress that no previous knowledge about the physics of such a receiver is assumed, but only a \textit{reward} is given to the agent if bit decoding (either $0$ or $1$, standing for $\proj{\alpha_0}$ or $\proj{\alpha_1}$) happens to be correct.

The model-free learning Dolinar-like receivers approach is explained in ~\ref{sssec:4_model_free_approach_intro}.
While we observe that our agent attains an \textit{empirical} success rate that equals the best attainable for the receiver at hand (\textit{e.g.} fixed $L=2$), in ~\ref{sssec:7_rl_dolinar_plus_bandit} we borrow tools from bandit problems, introduced in Sec.~\ref{sssec:15_badit_problem}, and improve agent's learning behaviour. In particular, a clever approach for discovering new configurations given past observations of measurement outcomes and rewards. A plethora of methods, ideas and algorithms can be brought into attention so to enhance our agent, and in ~\ref{sssec:37_bis_enchancements} we briefly review some of them, with partial results (and also failiures) when using them to tackle the coherent-state discrimination problem. In particular, this matter was the subject of a Degree thesis ~\cite{JoseTFG}.

One of the advantages of going for a model-free approach is that of noise-reisiliance: the agent learns a convinient strategy for the scenario at hand, since it is not biased towards the noise-less case. In Sec.~\ref{sssec:8_noise} we consider situations in which noise is present: from malfunctioning devices to unknown quantum channels acting between transmiter and receiver. In all of them we show succesful learning.

Finally, this chapter ends with an outlook and future work to be done, to be found in Sec.~\ref{sec:12_outlook}.
% %%%Helstrom
