The results presented in this Chapter stand as one of the first steps taken towards the reinforcement learning of quantum systems, in an experimentally-centered paradigm (note that this project was initiated back in 2018).

Here, we have introduced a reinforcement-learning agent that is able to calibrate a Dolinar-like receiver by only observing binary bits of information. Such receivers depend on a tree of conditional displacements that are applied to the incoming state, and the agent is asked to find the optimal configuration (namely, the set of displacements whose success probability is the highest). This is accomplished by repeating the discrimination experiment several times; by the end of each repetition the agent needs to guess for the underlying hypothesis and if such guess is correct, a reward of $1$ is given to her. This constitutes a big challenge, since there is a finite probability of not enjoying the reward while performing optimally (\textit{i.e.} using the best receiver's configuration). To this end, we have adapted the $Q$-learning algorithm to the task at hand, and also improved it by means of bandit methods. Our method shows succesful, in the sense that after a finite amount of repetitions the agent is able to achieve near-optimal calibrations. The usefulness of our approach has been showcased in wide range of non-ideal situations, where the presence of noise degrades the success probability of the receiver (and importantly, also modifies its optimal configuration). However, we showed that the agent can readily adapt to any of the situations considered.

The approach we have presented can readily be implemented as a proof of principle. While the original proposal of the receiver, \textit{i.e.} the time-domain Dolinar receiver, is slightly more experimentally friendly than the spatial-configuration one (though theoretically equivalent, as discussed in Sec.~\ref{ssec:tdol}), we observe that the application of reinforcement-learning algorithms in time-domain is still in its first steps, even theoretically~\cite{doya,cagata}. In this regard, conditional displacements can be done relatively fast using a Field Programmable Gate Arrays (FPGA), and we foresee a possible application of the finite receiver we considered $(L=2)$ that implements our reinforcement-learning method. In turn, this constitutes an on-going collaboration with the experimental group of Lorena Rebón in La Plata, Argentina.

Regarding the machine learning framework we have posed, there is certainly room for improvement. For example, the displacements (actions) were here considered to be discrete, whereas their nature is actually continuous. This issue has partially been tackled with a degree thesis~\cite{joseTFG}, in which we have studied the usage of neural networks (NNs). Here, we leverage the predictive power of NNs to interpolate state-action value functions to points in phase space $(s,a)$ that have not been visited (and thus we lack their estimate). However, training NNs with a low amount of data is a subtle matter. In the degree thesis, we have studied the performance of algorithms inspired by the Deep Deterministic Policy Gradient algorithm~\cite{DDPGpaper}, where both policy and state-action value functions are predicted by deep neural networks. However, we were not able to go further than $L=1$ unless the number of experiments were ridiculously high, and more work needs to be done in this respect. A possible circumvent to this issue is the usage of pre-trained generative models, to have a clever ansatz for the state-action value function profile.

Overall, many doors remain open to deeper understand the reachability of reinforcement-learning algorithms to model-free discrminating between quantum states in an agnostic way.

\bigskip

While the reinforcement learning of quantum systems is a relatively young research direction, it has arguably positioned as a state of the art technique in many branches of science. In turn, approaches based in deep reinforcement learning have recently lead to breakthroughs in protein-folding problem~\cite{Jumper2021}, quantum sensing~\cite{Schuff_2020}, quantum communications~\cite{PScomm}, quantum computing~\cite{Cerezo2021,foselgoogleRL} and more~\cite{Carleo2016,Torlai2016,VanNieuwenburg2017,Carrasquilla2017,Torlai2018,Melnikov2018,Fosel2018,Wallnofer2019,Bukov2017,Niu2019}.

Whether such a human-computer collaboration will remain as a state-of-the-art paradigm for the discovery of novel protocols in science and technology, or is a stepping stone towards a deeper understanding of knowledge is an open and existing question.
