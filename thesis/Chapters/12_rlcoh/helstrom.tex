In this case, the optimization over all POVMs leads to a Helstrom bound in Eq.~\ref{eq:helstrom_pure} that reads
\begin{align}\label{eq:hel_coh}
P_{s}^{(hel)}(\alpha)=\max_{\mathcal{M}}P_{s}(\alpha,\mathcal{M})=\frac12\left(1+\sqrt{1-e^{-4\abs{\alpha}^{2}}}\right),
\end{align}
We note that the Helstrom probability tends to $1/2$ for $|\alpha|\rightarrow0$, i.e., the states become indistinguishable at very low energies. While the projection over a positive and negative part of $\Delta = \frac{1}{2}\big(\proj{\alpha_0}-\proj{\alpha_1}\big)$ results into a projection over cat-like states of the form $\ket{\alpha_0} + \ket{\alpha_1}$~\cite{jap_cat}. While the preparation of such a states is currently beyond the scope in quantum optics laboratories, an implementation of such a projection can be carried out with linear optics and photon-detectors~\cite{takeoka, dolinar}. This constitutes the so-called Dolinar receiver, which is sequential by nature and very interesting from the fundamental point of view. In turn, it has been proven that local, sequential strategies are assymptotically optimal for binary (pure) state discirmination~\ref{acin}, where a Bayesian approach is considered at each step, where each new piece of information steers the guess towards the correct one.

% Let us discuss another approach to computing the Helstrom bound, which will be used in Sec.~\ref{ssec:8_noise} when dealing with compound channels. In particular, such a case is similar as considering the discrimination of more than two hypothesis, and we will now outline a method to find the Helstrom lower bound.
%
%




Before jumping to the Dolinar receiver, we will briefly consider how Gaussian measurements deal with the binary coherent-state discirmination problem.
