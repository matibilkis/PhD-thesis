\chapter*{Abstract}
The success in the experimental control of quantum systems, along with the advent of artificial intelligence, raises the question of where do these worlds meet. In this thesis, a series of frameworks is proposed, in which a machine-learning agent sequentially interacts with a quantum system in order to learn control strategies. Such frameworks are interconnected with each other by the degree of knowledge the agent possesses on the setting at hand.

Our journey begins by model-free learning of long-distance quantum communication protocols, continues towards an intermediate-stage of awareness in which NISQ-computing problems are tackled, and finally lands on the fully-aware terrain, where statistical-inference problems are considered for continuously-monitored systems.

Specifically, we depart from a completely agnostic assumption, where a reinforcement learning agent is posed to the task of calibrating a quantum receiver that decodes information out of a continuous-variable system. From here, some light is shed into the learning problem, and in the second framework we consider how to design useful quantum circuits for currently-available quantum computers. To this end, a semi-agnostic algorithm that jointly optimizes both the structure and the parameters of the circuit is introduced. Finally, we consider how open quantum dynamics can be inferred by an agent which is continuously-monitoring a system. Here, the structure of the quantum evolution is assumed to be known, and the agent needs to learn which is the underlying hypothesis that describes systems’ dynamics out of a noisy measurement signal.

The methods developed in this thesis can potentially be applied to real-life scenarios, and represent a step forward towards the constitution of quantum machine learning.
