\chapter*{Resum}
L’èxit en el control experimental dels sistemes quàntics, juntament amb la presència quotidiana de diferents intel·ligències artificials, planteja la pregunta d’on es creuen aquests mons. En aquesta tesi es proposa una sèrie de marcs en què un agent d’aprenentatge automàtic interactua seqüencialment amb un sistema quàntic per tal d’aprendre estratègies de control. Aquests marcs estan interconnectats entre si pel grau de coneixement l’agent té sobre l’escenari en qüestió.

El nostre viatge comença amb l’aprenentatge sense model de protocols de comunicació quàntica sobre llarga distància, continua cap a una etapa intermèdia de consciència en què s’aborden problemes de computació quàntica tipus NISQ i finalment aterra al terreny totalment conscient, on es consideren problemes d’inferència estadística en sistemes quàntics contínuament monitoritzats.

Específicament, partim d’un escenari completament agnòstic on un agent d’aprenentatge per reforç s’enfronta a la tasca de calibrar un receptor quàntic que descodifica informació clàssica d’un sistema quàntic de variable contínua. Partint d’aquest context, llancem una mica de llum sobre el problema de l’aprenentatge, i el segon escenari consisteix a dissenyar circuits quàntics útils per als ordinadors quàntics actualment disponibles. Per això, introduïm un algorisme semi-agnòstic que optimitza de manera conjunta tant l’estructura com els paràmetres del circuit. Finalment, considerem com la dinàmica d’un sistema quàntic pot ser inferida per un agent que està monitoritzant contínuament aquest sistema. Aquí, assumim que l’estructura de la dinàmica és coneguda per l’agent, que necessita aprendre quina és la hipòtesi subjacent que descriu l’evolució a partir d’un senyal de mesura sorollós.

Els mètodes desenvolupats en aquesta tesi estan preparats per poder ser aplicats a escenaris de la vida real i representen un pas endavant cap a la constitució de l’aprenentatge automàtic quàntic.
\setcounter{page}{1}
