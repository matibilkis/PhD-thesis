\chapter*{Resumen}
El control experimental exitoso de ciertos sistemas cuánticos, junto con la presencia de la inteligencia artificial, plantea la pregunta de dónde estos mundos se encuentran. En esta tesis proponemos una serie de marcos donde agente de aprendizaje automático interactúa secuencialmente con un sistema cuántico para aprender estrategias de control. Los marcos están interconectados entre sí por el grado de conocimiento el agente posee sobre el escenario en cuestión.

Nuestro viaje comienza con el aprendizaje sin modelo para protocolos de comunicación cuántica, continúa hacia una etapa intermedia de conciencia donde se abordan problemas de computación cuántica, y finalmente aterriza en el terreno totalmente consciente, considerando problemas de inferencia estadística en sistemas continuamente monitoreados.

En particular, partimos de una escenario agnóstico, donde un agente de aprendizaje por refuerzo se enfrenta a la tarea de calibrar un receptor cuántico que decodifica estados coherentes. Continuamos arrojando algo de luz sobre el grado de conciencia, y nuestro segundo escenario consiste en el diseño de circuitos cuánticos útiles para dispositivos actualmente disponibles. Para ello, introducimos un algoritmo semi-agnóstico que optimiza conjuntamente la estructura y los parámetros del circuito. Finalmente, estudiamos cómo inferir la dinámica de un sistema a partir de su monitoreando continuo. Aquí, asumimos que la estructura de la dinámica es conocida por el agente, quien necesita aprender cuál es la hipótesis subyacente que describe la evolución, a partir de la señal de medidas ruidosas.

Los métodos desarrollados en esta tesis están listos para ser aplicados a escenarios de la vida real y representan un paso adelante hacia la constitución del aprendizaje automático cuántico.
